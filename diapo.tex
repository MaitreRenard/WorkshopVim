\documentclass[10pt]{beamer}

%Paquets d'écriture et de compatibilité française
\usepackage[T1]{fontenc} 
\usepackage{lmodern}

%Paquets d'ajouts esthétiques
% Pour pouvoir aligner les images correctement
\usepackage{graphbox}
% Pour avoir les tableaux
\usepackage{tabularx}
% Pour faire des cellules de tableau
\usepackage{makecell}
% Gestion des images
\usepackage{graphicx}
% Gestion du thème
\usetheme{metropolis}
%\setbeameroption{show only notes}
%\setbeamercolor{progress bar}{
%	use=alerted text,
%	fg=blue,
%	bg=alerted text.fg!50!black!30
%}

%\setbeamertemplate{frame footer}{\footnotesize\insertsection}
\setbeamerfont{alerted text}{series=\bfseries,size=\huge}

\metroset{block=fill,progressbar=frametitle}

\makeatletter
\setlength{\metropolis@titleseparator@linewidth}{1.5pt}
\setlength{\metropolis@progressonsectionpage@linewidth}{1.5pt}
\setlength{\metropolis@progressinheadfoot@linewidth}{1.5pt}

\setbeamertemplate{title page}{
	\begin{minipage}[b][\paperheight]{\textwidth}
		\ifx\inserttitlegraphic\@empty\else\usebeamertemplate*{title graphic}\fi
		\vfill%
		\ifx\inserttitle\@empty\else\usebeamertemplate*{title}\fi
		\ifx\insertsubtitle\@empty\else\usebeamertemplate*{subtitle}\fi
		\usebeamertemplate*{title separator}
		\vspace*{5mm}

		\begin{tabularx}{\textwidth}{Xr}
			\makecell[l]{
				\ifx\beamer@shortauthor\@empty\else\usebeamertemplate*{author}\fi\\
				\ifx\insertinstitute\@empty\else\usebeamertemplate*{institute}\fi
			}
		\end{tabularx}

		\vfill
		\vspace*{1mm}
	\end{minipage}
}

\setbeamertemplate{title}{
	\linespread{1.0}%
	\inserttitle%
	\par%
	\vspace*{0.5em}
}

\setbeamertemplate{subtitle}{
	\insertsubtitle%
	\par%
	\vspace*{0.5em}
}

\setbeamertemplate{author}{
	%\linespread{1.0}%
	\large{\insertauthor}%
	\par%
	\vspace*{0.5em}
}
\makeatother


\title{Workshop VIM}
\subtitle{Ou comment passer pour un puriste}
\author{MasterFox}

\begin{document}
	  \maketitle

		\section{Vi : Les origines}
			\begin{frame}{\textbf{C'est l'histoire de la Vi}}

				\centering{Vi \rightarrow VIm \rightarrow NeoVim}

				%Tout commença en 1976
				\\
				\vspace{20pt}

				\raisebox{-0.5\height}{\includegraphics[height=42]{img/logo-vim.png}}
				\hspace{10pt}
				\raisebox{-0.5\height}{\includegraphics[height=42]{img/logo-neovim.png}}

			\end{frame}

		\section{Les bases}
			\begin{frame}{Déplacements relatifs}
				\includegraphics{img/control-keys.png}
				w, b
				), (
				\}, \{

			\end{frame}

			\begin{frame}{Déplacements absolus}
				\%
				G

				H, L, M
			\end{frame}

		\begin{frame}[standout]
			Jouons à un PETIT JEU !\\
			\vspace{10pt}
			\includegraphics[width=48]{img/edit-vimrc.png}\\
			\vspace{10pt}
			\includegraphics[width=128]{img/harder.png}
		\end{frame}

		\section{Les registres}
			\begin{frame}{Keskessé}
			\end{frame}

			\begin{frame}{Les marques}
				\center
				\includegraphics[height=100]{img/markers.png}
			\end{frame}

			\begin{frame}{Les macros}
				\center
				\includegraphics[height=100]{img/markers.png}
			\end{frame}

		\section{Le début des trucs bien stylés}
		\section{Le \textit{R I C E}}

	\begin{appendix}

		\begin{frame}[standout]
			Merci d'avoir écouté !
		\end{frame}

		\begin{frame}
			\frametitle{Sources}
			\begin{itemize}
				%Documentation
				\item \href{https://neovim.io/}{Neovim}
				\item \href{https://www.vim.org}{Vim}
			\end{itemize}
		\end{frame}

		\begin{frame}[standout]
			Des questions ?
		\end{frame}

	\end{appendix}
\end{document}
